\chapter{Vector template example}
\label{appendix:vectemplate}

The following listings contain a sample implementation of vector templates as they have been described in Section \ref{section:IR}. The examples use int as a base type, however the code is pretty much the same for the other base types. Because of this, the templates are generated automatically (which also explains the lack of comments in the code).

The base template is shown in Listing \ref{lst:vectempl}.
\lstinputlisting[
	label=lst:vectempl, 
	caption=Sample vector template, 
	lastline=295]
{src/code/vectemplate.cpp}

Listing \ref{lst:scalartempl} contains a specialization of the template from listing \ref{lst:vectempl} - its purpose is to allow a smooth interaction between vectors and scalar values of the same base type. It achieves this by overloading the implicit conversion operator from the vector to the corresponding base type.
\lstinputlisting[
	label=lst:scalartempl, 
	caption=Template specialization for scalars, 
	firstline=297, 
	lastline=326]
{src/code/vectemplate.cpp}

Another template specialization is presented in Listing \ref{lst:dummytempl}. This is a dummy specialization, necessary for the implementation of the comma operator hack.
\lstinputlisting[
	label=lst:dummytempl, 
	caption=Dummy template specialization, 
	firstline=328]
{src/code/vectemplate.cpp}